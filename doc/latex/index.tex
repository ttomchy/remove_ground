\href{https://travis-ci.org/niosus/depth_clustering}{\tt !\mbox{[}Build Status\mbox{]}\mbox{[}travis-\/img\mbox{]}} \href{https://www.codacy.com/app/zabugr/depth_clustering?utm_source=github.com&amp;utm_medium=referral&amp;utm_content=niosus/depth_clustering&amp;utm_campaign=Badge_Grade}{\tt !\mbox{[}Codacy Badge\mbox{]}\mbox{[}codacy-\/img\mbox{]}} \href{https://coveralls.io/github/niosus/depth_clustering?branch=master}{\tt !\mbox{[}Coverage Status\mbox{]}\mbox{[}coveralls-\/img\mbox{]}}

This is a fast and robust algorithm to segment point clouds taken with Velodyne sensor into objects. It works with all available Velodyne sensors, i.\-e. 16, 32 and 64 beam ones.

Check out a video that shows all objects which have a bounding box of less than 10 squared meters\-: \href{https://www.youtube.com/watch?v=mi-Z__B1yyE}{\tt !\mbox{[}Segmentation illustration\mbox{]}(http\-://img.\-youtube.\-com/vi/mi-\/\-Z\-\_\-\-\_\-\-B1yy\-E/0.\-jpg)}

\subsection*{How to build?}

\subsubsection*{Prerequisites}


\begin{DoxyItemize}
\item Catkin.
\item Open\-C\-V\-: {\ttfamily sudo apt-\/get install libopencv-\/dev}
\item Q\-G\-L\-Viewer\-: {\ttfamily sudo apt-\/get install libqglviewer-\/dev}
\item Qt (4 or 5 depending on system)\-:
\begin{DoxyItemize}
\item {\bfseries Ubuntu 14.\-04\-:} {\ttfamily sudo apt-\/get install libqt4-\/dev}
\item {\bfseries Ubuntu 16.\-04\-:} {\ttfamily sudo apt-\/get install libqt5-\/dev}
\end{DoxyItemize}
\item (optional) P\-C\-L -\/ needed for saving clouds to disk
\item (optional) R\-O\-S -\/ needed for subscribing to topics
\end{DoxyItemize}

\subsubsection*{Build script}

This is a catkin package. So we assume that the code is in a catkin workspace and C\-Make knows about the existence of Catkin. Then you can build it from the project folder\-:


\begin{DoxyItemize}
\item {\ttfamily mkdir build}
\item {\ttfamily cd build}
\item {\ttfamily cmake ..}
\item {\ttfamily make -\/j4}
\item (optional) {\ttfamily ctest -\/\-V\-V}
\end{DoxyItemize}

It can also be built with {\ttfamily catkin\-\_\-tools} if the code is inside catkin workspace\-:
\begin{DoxyItemize}
\item {\ttfamily catkin build depth\-\_\-clustering}
\end{DoxyItemize}

P.\-S. in case you don't use {\ttfamily catkin build} you \href{https://catkin-tools.readthedocs.io/en/latest/installing.html}{\tt should}. Install it by {\ttfamily sudo pip install catkin\-\_\-tools}.

\subsection*{How to run?}

See \href{examples/}{\tt examples}. There are R\-O\-S nodes as well as standalone binaries. Examples include showing axis oriented bounding boxes around found objects (these start with {\ttfamily show\-\_\-objects\-\_\-} prefix) as well as a node to save all segments to disk. The examples should be easy to tweak for your needs.

\subsection*{Run on real world data}

Go to folder with binaries\-: ``` cd $<$path\-\_\-to\-\_\-project$>$/build/devel/lib/depth\-\_\-clustering ```

\paragraph*{Frank Moosmann's \char`\"{}\-Velodyne S\-L\-A\-M\char`\"{} Dataset}

Get the data\-: ``` mkdir data/; wget \href{http://www.mrt.kit.edu/z/publ/download/velodyneslam/data/scenario1.zip}{\tt http\-://www.\-mrt.\-kit.\-edu/z/publ/download/velodyneslam/data/scenario1.\-zip} -\/\-O data/moosmann.\-zip; unzip data/moosmann.\-zip -\/d data/; rm data/moosmann.\-zip ```

Run a binary to show detected objects\-: ``` ./show\-\_\-objects\-\_\-moosmann --path data/scenario1/ ```

\paragraph*{Other data}

There are also examples on how to run the processing on K\-I\-T\-T\-I data and on R\-O\-S input. Follow the {\ttfamily -\/-\/help} output of each of the examples for more details.

\subsection*{Documentation}

You should be able to get Doxygen documentation by running\-: ``` cd doc/ doxygen Doxyfile.\-conf ```

\subsection*{Related publications}

Please cite related papers if you use this code\-:

``` \{bogoslavskyi16iros, title = \{Fast Range Image-\/\-Based Segmentation of Sparse 3\-D Laser Scans for Online Operation\}, author = \{I. Bogoslavskyi and C. Stachniss\}, booktitle = \{Proc. of The International Conference on Intelligent Robots and Systems (I\-R\-O\-S)\}, year = \{2016\}, url = \{\href{http://www.ipb.uni-bonn.de/pdfs/bogoslavskyi16iros.pdf}{\tt http\-://www.\-ipb.\-uni-\/bonn.\-de/pdfs/bogoslavskyi16iros.\-pdf}\} \} ```

``` \{bogoslavskyi17pfg, title = \{Efficient Online Segmentation for Sparse 3\-D Laser Scans\}, author = \{I. Bogoslavskyi and C. Stachniss\}, journal = \{P\-F\-G -- Journal of Photogrammetry, Remote Sensing and Geoinformation Science\}, year = \{2017\}, pages = \{1--12\}, url = \{\href{https://link.springer.com/article/10.1007%2Fs41064-016-0003-y}{\tt https\-://link.\-springer.\-com/article/10.\-1007\%2\-Fs41064-\/016-\/0003-\/y}\}, \} ``` 